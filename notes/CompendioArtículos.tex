\documentclass[11pt,letterpaper]{article}

\usepackage[utf8]{inputenc} \usepackage[spanish]{babel} \usepackage{amsmath}
\usepackage{amsfonts} \usepackage{amssymb} \usepackage{graphicx} \usepackage{hyperref}
\usepackage{mhchem} \usepackage{nameref} \usepackage{caption} \usepackage{subcaption}
\usepackage[left=1.5cm,top=1.5cm,right=1.5cm,bottom=2.5cm]{geometry} \usepackage{color}
\usepackage{epstopdf} \providecommand{\e}[1]{\ensuremath{\times 10^{#1}}}
\usepackage{wrapfig} \newcommand{\ncd}{\newcommand} \ncd{\mrm} {\mathrm} \ncd{\beq}
{\begin{equation}} \ncd{\eeq} {\end{equation}} \def\d{{\rm d}} \def\D{{\rm D}}
\def\f{{\rm f}} \def\g{{\rm g}} \def\r{\mathbf{r}} \def\p{\mathbf{p}} \def\q{\mathbf{q}}
\newcommand{\avg}[1]{\left< #1 \right>} % for average
\newtheorem{prop}{Proposición}[section]
\newtheorem{teor}[prop]{Teorema}


\begin{document}
\title{Compendio artículos}
\maketitle

\begin{abstract}
Estas notas están hechas con el objeto de hacer un breve compendio de los principales trabajos relacionados con la simulación del Billar de Discos Duros.
\end{abstract}

\section*{Colloquium: Phononics: Manipulating heat flow with electronic
analogs and beyond; Li, Nianbei et al., Rev. Mod. Phys. 2012}

Se propone la fonónica como ``disciplina'' análoga a la electrónica. 

Esencial:
\begin{itemize}
\item  Concepto de fonón (cadenas (an)armónicas).
\item  Espectro de potencias. 
\end{itemize}

\section*{Anomalous Heat Diffusion; Li, Nianbei et al.; PRL, 2014.}

Objetivo
\begin{itemize}
\item Mostrar que el comportamiento anómalo en la conductividad térmica puede ser unívocamente realcionada a la dispersión de energía anómala difusiva en ¡¿fases sólidas?!
\end{itemize}

Esencial
\begin{itemize}
\item En el caso del transporte anómalo la conductividad térmica no relaciona directamente la densidad de flujo de energía local con el gradiente de temperatura local.
\item La variancia se representa como $\avg{\Delta x^2(t)}_E$ y puede tomar valores negativos. Los autores dicen que esto es muestra de que es la varianza para la distribución del exceso de energía y no para los desplazamientos de las posiciones de las partículas.
\item En la derivación de la relación entre la función de autocorrelación y la variación de la enrgía local no entra ninguna temperatua local de equilibrio.
\end{itemize}

Resultado.
\begin{itemize}
\item Ecuación de movimiento para la varianza que da cuenta de la dispersión de enrgía.
\end{itemize}

\end{document}